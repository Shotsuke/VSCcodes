
\documentclass[UTF8]{ctexart}

\usepackage[a4paper, total={6.5in, 9.5in}]{geometry}

\begin{document}
%标题
\section*{
  \begin{center}
    \huge
    信息与计算科学导论\\
    221502023 \  沈硕 \  第五次作业\\
  \end{center}
 }
%第一题
\subsection*{
  \begin{flushleft}
    Problem 1:
    Suppose that p(L) is a polynomial in L,
    and that p(L) annihilates function T(n).
    Prove that
    $T(n) = \sum_{i=1}^k f_\mathrm{i}(n)r_\mathrm{i}^n$,
    where k is a constant,
    $f_\mathrm{1}(n), . . . , f_\mathrm{k}(n)$ are polynomials in n,
    and $r_\mathrm{1}, . . . , r_\mathrm{k}$ are all constants.
    \\
  \end{flushleft}
}
%第一题解答
\begin{flushleft}
  Solution: 不妨设:

  \begin{center}
    $p(L) = a_0 + a_1L + a_2L^2 + \ldots + a_mL^m$
  \end{center}

  将$L$带入$T(n)$可以得到:

  \begin{center}
    $0=p(L)T(n)= a_0T(n) + a_1LT(n) + a_2L^2T(n) + \ldots +a_mL^mT(n)$
  \end{center}

  那么,p(L)=0的m个解即为T(n)的m个特征根。
  我们记p(L)=0的k个不重复的解为$r_\mathrm{1}, . . . , r_\mathrm{k}$时,
  就得到了

  \begin{center}
    $T(n) = \sum_{i=1}^k f_\mathrm{i}(n)r_\mathrm{i}^n$
  \end{center}

  其中第i个解$r_i$前的$f_i(n)$的最高次数表示了其重根数$-1$。

\end{flushleft}
%第二题
\subsection*{
  \begin{flushleft}
    Problem 2:
    (Based on an  Idea from IMO-1986-2)
    In a plane there is a regular triangle ABC with side length 1.
    Two frogs, Alice and Bob, are jumping.
    Let $A_t, B_t$ be the locations of Alice and Bob, respectively, at time t. \\
    • $A_0 = B_0 = C.$ \\
    • $A_1 = A.$ \\
    • For t ≥ 2, rotate the line segment $BA_\mathrm{t-1}\ 45$\textdegree
    clockwise about point B to get get $A_t^{'}$;
    rotate the line segment $BA_\mathrm{t-2} 45$\textdegree
    clockwise about point B to get get $A_\mathrm{t}^{''}$;
    the midpoint of $A_t^{'}A_t^{''}$ is $A_t$. \\
    • For t ≥ 1, rotate the line $CB_\mathrm{t-1}\ 45$\textdegree
    clockwise about point C to get the new line $CB_t$.
    The length of line segment $CB_t$ is twice that of line segment $CB_\mathrm{t-1}$.
    Find the distance of $B_t$ from $A_t$.\\
  \end{flushleft}
}
%第二题解答
\begin{flushleft}
  Solution: I give up.\\
\end{flushleft}
%第三题
\subsection*{
  \begin{flushleft}
    Problem 3:
    Define a difference operator
    $\Delta: \Delta = L - 1$
    , i.e.,
    $\Delta T(n) = T(n + 1) - T(n).$
    Define a sum operator
    $\sum: \sum T = S + c$
    where
    $\Delta S = T$
    and c is a constant.
    Prove that for all functions f, g defined on positive integers,
    $\sum f\Delta g = fg - \sum Lg\Delta f.$\\
  \end{flushleft}
}
%第三题解答
\begin{flushleft}
  Solution:
  \begin{center}
    $\Delta (f(n)g(n)) = f(n+1)g(n+1) - f(n)g(n) =
      f(n+1)g(n+1) + f(n)g(n+1) - f(n)g(n+1) - f(n)g(n)$\\
    $\Delta f(n)g(n) = g(n+1)\Delta f(n) + f(n)\Delta g(n)$\\
    $\Delta f(n)g(n) = Lg(n)\Delta f(n) + f(n)\Delta g(n)$\\
  \end{center}
  对两边同时作$\sum $即得之。
\end{flushleft}
%第四题
\subsection*{
  \begin{flushleft}
    Problem 4:
    The first case of Theorem 6 uses the following condition:
    $f(n) = \Theta (n^c) $
    where
    $c < log_ba$.
    In the literature, you may find an alternative condition for this case:
    $f(n) = O(n^{log_ba-\epsilon})$
    for some
    $\epsilon > 0$.
    Prove using these two conditions in the theorem are equivalent.
  \end{flushleft}
}
%第四题解答
\begin{flushleft}
  Solution:
  \begin{center}
    \begin{large}
      $T(n) = \sum_{i=0}^{log_bn} a^i f(\frac{n}{b^i}) + O(n^{log_ba})$\\
      $\leq\ \sum_{i=0}^{log_bn} a^i M (\frac{n}{b^i})^{log_ba - \epsilon} + O(n^{log_ba}) $\\
      $=\ M n^{lob_ba - \epsilon} \sum_{i=0}^{log_bn} a^i b^{-ilog_ba + i\epsilon} + O(n^{log_ba})$\\
      $=\ M n^{log_ba - \epsilon} \sum_{i=0}^{log_bn} b^{i\epsilon} + O(n^{log_ba})$\\
      $=\ M n^{log_ba - \epsilon} \frac{b^\epsilon (log_bn + 1) -1}{b^\epsilon - 1} + O(n^{log_ba})$\\
      $=\ M n^{log_ba - \epsilon} \frac{(nb)^\epsilon - 1}{b^\epsilon - 1} + O(n^{log_ba})$\\
      $\leq\ M n^{log_ba - \epsilon} \frac{(nb)^\epsilon}{b^\epsilon - 1} + O(n^{log_ba})$\\
      $=\ M n^{log_ba} \frac{b^\epsilon}{b^\epsilon - 1} + O(n^{log_ba})$\\
      $=\ O(n^{log_ba})$\\
    \end{large}
  \end{center}
  则得证。此处的
  $f(n) = O(n^{log_ba-\epsilon})$
  等价于
  $f(n) = \Theta (n^c)$。
  一方面,两者均小于
  $log_ba$,这保证了其增长速度不超过$\Theta n^{log_ba}$,从而得到相同的结论;
  另一方面,可以通过调整$\epsilon$的值来使得二者相同,
  即使是选择了较小的$\epsilon$而使得f(n)的增长速度比实际的要更快,
  这也只是说明其增速慢于$\Theta (n^{log_ba})$,对其粗略的估计也无妨于结论。\\
\end{flushleft}
%第五题
\subsection*{
  \begin{flushleft}
    Problem 5:
    Find a recurrence
    $T(n) = aT(\frac{n}{b}) + f(n)$
    such that
    $f(n) = \Theta (n^c)$ where $c > log_ba$,
    but f(n) does not meet the regularity condition.
  \end{flushleft}
}
%第五题解答
\begin{flushleft}
  Solution :
  我们只需要破坏正则条件即可。例如:
  \begin{center}
    \begin{large}
      $f(n) = n^c\sin n$\\
    \end{large}
    其的确是$O(n^c)$,但是没法保证
    \begin{center}
      $af(\frac{n}{b}) \leq df(n)$ for some
      $d < 1$ and sufficiently large n\\
    \end{center}
  \end{center}

\end{flushleft}
\end{document}