\documentclass[UTF8]{ctexart}
\usepackage{setspace}
\usepackage{amssymb}
\usepackage[a4paper, total={6.5in, 9.5in}]{geometry}

\begin{document}
%标题
\section*{
  \begin{center}
      \huge
      信息与计算科学导论\\
      Number Theory\\
      221502023 \  沈硕 \  第四次作业\\
  \end{center}
 }

%Problem 38
\subsection*{
    \begin{flushleft}
        \Large
        Problem 38
            [Difficulty Estimate=2.4]
        Prove Proposition 52 without using induction.\\
        Proposition 52:
        For any distinct odd numbers p, q > 1,
        suppose f is the permutation on $Z_{pq}$ that maps
        aq + b to a + bp for every $a \in Z_p, b \in Z_q$.
        Then, f has the same parity as $\frac{(p-1)(q-1)}{4}$.
    \end{flushleft}
}%Solution to Problem 38
\begin{flushleft}
    \Large

\end{flushleft}

%Problem 39
\subsection*{
    \begin{flushleft}
        \Large
        Problem 39
            [Difficulty Estimate=1.2]
        Prove Propositions 53 and 54.\\
        Proposition 53:
        For any positive odd integers n, m and any integers a, b,
        the Jacobi symbol $(\frac{ab}{n}) =(\frac{a}{n})(\frac{b}{n})$;
        the Jacobi symbol $(\frac{a}{nm})=(\frac{a}{n})(\frac{a}{m})$.
        \\Proposition 54 :
        For any positive odd integers n,
        the Jacobi symbol $(\frac{-1}{n}) = 1$ if $n \equiv 1\ (mod\ 4)$;
        $(\frac{-1}{n})=-1$ if $n \equiv 3\ (mod\ 4)$;
        the Jacobi symbol $(\frac{2}{n}) = 1$ if $n \equiv \pm 1\ (mod\ 8)$;
        $(\frac{2}{n}) = -1$ if $n \equiv \pm 3\ (mod\ 8)$.
    \end{flushleft}
}%Solution to Problem 39
\begin{flushleft}
    \Large
    Solution to Proposition 53:\\
    Denote n as $p_1^{k_1}\cdots p_l^{k_l}$,
    where $p_1 \cdots p_k$ are all odd primes.\\
    $\frac{ab}{n} = (\frac{ab}{p_1})^{k_1} \cdots (\frac{ab}{p_l})^{k_l}$,
    And as the $(\frac{ab}{p_i})^{k_i}$ can also be a Legender symbol,
    so it can be seperated as
    $((\frac{a}{p_i})(\frac{b}{p_i}))^{k_i}$.
    Apply this to all $i = 1,2,\cdots ,l$ can prove the former one.\\
    For the latter one, denote m as $q_1^{i_1}\cdots q_j^{i_j}$,
    then by the definition of Jacobi symbol we can similarly reach the proposition.\\
    Solution to Proposition 54:\\
    $(\frac{-1}{n}) = \prod_{i=1}^{l}(\frac{-1}{x_i})$, $n = x_1 x_2 \cdots x_l$,
    $\Rightarrow (-1)^{\frac{\prod_{i=1}^{l} x_i -1}{2}} = \prod_{i=1}^{l}(-1)^{\frac{x_i-1}{2}}$.
    So $(\frac{-1}{n}) = 1 \Leftrightarrow$ the number of $x_i \equiv 3\ (mod\ 4)$ is even.\\
    $\Leftrightarrow n \equiv 1\ (mod\ 4)$.
    Similar proof to $(\frac{-1}{n})=-1$ if $n \equiv 3\ (mod\ 4)$.\\
    If p is a prime, then $(\frac{2}{p}) = (-1)^{\frac{p^2-1}{8}}$.
    Then $(\frac{2}{n}) = (-1)^{\Sigma_{i=1}^{l}k_i\frac{p_i^2-1}{8}}$\\
    So $(\frac{2}{n}) = 1 \Leftrightarrow$
    the number of $p_i \equiv \pm 3\ (mod\ 8)$ is even.
    $\Leftrightarrow n \equiv 1\ (mod\ 4)$.
    Similar proof to $(\frac{2}{n}) = -1$ if $n \equiv \pm 3\ (mod\ 8)$.
\end{flushleft}

%Problem 40
\subsection*{
    \begin{flushleft}
        \Large
        Problem 40
            [Difficulty Estimate=1.2]
        Read another proof of Gauss quadratic reciprocity.
        For example, you may consider reading the original proof by Gauss,
        or the famous alternative proof by Einstein.
    \end{flushleft}
}%Solution to Problem 40
\begin{flushleft}
    \Large
    Solution: I have read it..??
\end{flushleft}

%Problem 41
\subsection*{
    \begin{flushleft}
        \Large
        Problem 41
            [Difficulty Estimate=1.8]
        Suppose a is a positive integer but not a perfect square
        (i.e., not equal to the square of any integer).
        Prove that there exist infinitely many primes p such that
        $(\frac{a}{p} = -1)$
    \end{flushleft}
}
\end{document}