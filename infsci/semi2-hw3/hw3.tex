\documentclass[UTF8]{ctexart}
\usepackage{setspace}
\usepackage{amssymb}
\usepackage[a4paper, total={6.5in, 9.5in}]{geometry}

\begin{document}
%标题
\section*{
  \begin{center}
      \huge
      信息与计算科学导论\\
      Number Theory\\
      221502023 \  沈硕 \  第三次作业\\
  \end{center}
 }

%Problem 24
\subsection*{
    \begin{flushleft}
        \Large
        Problem 24:
        (Adapted from Problems 777 and 778, [8])
        [Difficulty Estimate=1.3]
        Prove the two propositions below,
        both of which are about Euler's Totient Function.\\
        (1) (Recall that a perfect square is simply the square of an integer.)
        There are infinitely many positive
        integers n such that $n + \phi (n)$ is a perfect square.\\
        (2) For $r \geq 1$, there is no positive integer n such that $\phi (n) = 2 · 7^r$\\
    \end{flushleft}
}%Solution to Problem 24
\begin{flushleft}
    \Large
    Proof: (1) Note that $n = 5$ is a valid number.\\
    Consider $n = p^{2k+1}$.
    If $p + \phi (p) = 2p - 1$ is a perfect square $n^2$,
    then $p^{2k+1} + \phi (p^{2k+1}) = p^{2k+1} + p^{2k}(p-1) = (p^k m)^2$
    is also a perfect square.\\
    Immediately $5, 5^3, 5^5, \cdots$ is a valid infinite sequence due to $k \in N^+$.\\
    (2) Each n can be denoted by $n = p_1^{k_1}p_2^{k_2}\cdots p_m^{k_m}$,
    where $p_1, p_2, \cdots, p_m$ are all primes,
    and $k_1, k_2,\cdots , k_m$ are all positive integers.\\
    Now apply this to $\phi (n) = 2 · 7^r$,
    we get $p_1^{k_1}p_2^{k_2}\cdots p_m^{k_m} = 2·7^r$.\\
    Since $r \geq 1$, then we must have $7|n$.\\
    WLOG, let $p_1 = 7$ and $k_1 \geq 2$.
    Then immediately we see 6 is a factor in the left hand side
    but not in the right hand side, i,e,
    $6\mid \phi (n)$ but $6 \nmid 2·7^r$,
    so there doesn't exist such n.
\end{flushleft}

%Problem 25
\subsection*{
    \begin{flushleft}
        \Large
        Problem 25:
        (Putnam 1997-B5) [Difficulty Estimate=2.2]
        Prove that for $n \geq 2$,
        $2^{2^{\cdots ^2}} (n\ terms) \equiv 2^{2^{\cdots ^2}} (n-1\ terms)$
        (mod n).
    \end{flushleft}
}%Solution to Problem 25
\begin{flushleft}
    \Large
    Give up.
\end{flushleft}

%Problem 26
\subsection*{
    \begin{flushleft}
        \Large
        Problem 26:
        (ARMO 2000-Grade 10-Day 1-Problem 1)
        [Difficulty Estimate=0.7]
        Evaluate
        \begin{center}
            \Large
            $\lfloor \frac{2^0}{3} \rfloor +
                \lfloor \frac{2^1}{3} \rfloor +
                \lfloor \frac{2^2}{3} \rfloor + \cdots +
                \lfloor \frac{2^{100}}{3} \rfloor$
        \end{center}
    \end{flushleft}
}%Solution to Problem 26
\begin{flushleft}
    \Large
    Solution:
    $2^{2k} \equiv 1 (mod\ 3), 2^{2k+1} \equiv 2 (mod\ 3)$,
    where $k \in N$.
    So the original formula
    \begin{center}
        \Large
        \doublespacing
        $= \frac{1}{3} (2^0 + 2^1 + \cdots + 2^{100}) - \frac{1}{3}(51 + 100)$\\
        $= \frac{1}{3}·\frac{1-2^{101}}{1-2} - \frac{152}{3}$\\
        $= \frac{2^{101} - 152}{3}$
    \end{center}
\end{flushleft}

%Problem 27
\subsection*{
    \begin{flushleft}
        \Large
        Problem 27:
        (S. Berlov, ARMO 2014-Grade 11-Day 2-Problem 1)
        [Difficulty Estimate=2.3]
        Call a natural number n good if for any natural divisor a of n,
        we have that a + 1 is also divisor of n + 1.
        Find all good natural numbers.
    \end{flushleft}
}%Solution to Problem 27
\begin{flushleft}
    \Large
    Solution:
    First, notice that 1 and all the prime numbers bigger than 2 are all good numbers.\\
    The next is to show other natural numbers are all not good numbers.\\
    If n is a good number, denote that $n = ab, where a,b \in N^+$.
    Then $a+1 \mid ab+1 \Rightarrow a+1 \mid b-1$.\\
    Similarly we can get $b+1 \mid a-1$, so combine these two inequations:
    $a+1 < b-1 \ and \ b+1 < \ a-1$
    we can immediately draw a contradiction unless a or b is 1.
\end{flushleft}

%Problem 28
\subsection*{
    \begin{flushleft}
        \Large
        Problem 28:
        [Difficulty Estimate=2.5]
        Write a complete proof for Example 48.\\
        Example 48:
        (IMO Shortlist 2005-N6)
        Let a, b be positive integers such that
        $b^n + n$ is a multiple of $a^n + n$
        for all positive integers n. Prove that a = b.
    \end{flushleft}
}%Solution to Problem 28
\begin{flushleft}
    \Large
    Solution:
    $a^n + n \mid b^n + n \Rightarrow a^n + n \mid b^n - a^n$,
    so if for some p, s.t. $a^n \equiv -n (mod\ p)$,
    then $b^n \equiv a^n$ holds.\\
    Assume $a \neq b$, then there exists a prime p, s.t. $p \nmid b-a$.\\
    Then there definitely exist a positive integer k, s.t. $a \equiv k-1 (mod\ p)$.\\
    After fixing k, we can let $n = k(p-1)+1$. And this is a counterexample.\\
    1. $a^n \equiv a^{k(p-1)+1} \equiv a \equiv k-1 \equiv -kp+k-1 \equiv -n (mod \ p)$.\\
    2. $a^n \equiv a, b^n\equiv b (mod \ p)$. Since $p \nmid b-a \Rightarrow b\not\equiv a (mod \ p)$,
    i,e, $b^n \not\equiv a^n (mod\ p)$, then we draw a contradiction.\\
    So there must be $a = b$.
\end{flushleft}

%Problem 29
\subsection*{
    \begin{flushleft}
        \Large
        Problem 29:
        [Difficulty Estimate=1.7]
        For what kind of odd primes p, is -3 a quadratic residue mod p?
        Prove your answer.
    \end{flushleft}
}%Solution to Problem 29
\begin{flushleft}
    \Large
    Solution:
    Of course 3 is not a valid odd prime number.\\
    Then we should think of  $p=12k+1 , 12k+5 , 12k+7 , 12k+11$ ,
    which definitely include all odd primes.\\
    Case 1: p = 12k + 1.\\
    \begin{center}
        \Large
        $\{ a \mid a \leq \frac{p-1}{2}, -3a\ mod\ p > \frac{p-1}{2} , a \in Z_p^+\}$ \\
        $ = \{ a \mid 1 \leq a \leq 2k\ or\ 4k+1 \leq a \leq 6k\ or\ 8k+1 \leq a \leq 10k\}$
    \end{center}
    Then this set has 6k elements, and by Gauss Lemma we can get that
    $-3 \in QR_p$.\\
    And similarly we can deal with the other 3 conditions.
    The conclution is that $-3 \in QR_p \iff p=12k+1 \ or \ 12k+7$.
\end{flushleft}

%Problem 30
\subsection*{
    \begin{flushleft}
        \Large
        Problem 30:
        (adapted from [9])
        [Difficulty Estimate=2.5]
        Suppose $m = 2^ap^b$, where p is an odd prime,
        and $a \leq 3$ and $b \leq 2$ are integers.
        What is $\Pi_{r^2 \equiv 1 (mod\ m)} r$?
        Prove your answer.
    \end{flushleft}
}%Solution to Problem 30
\begin{flushleft}
    \Large
    Give up.
\end{flushleft}

%Problem 31
\subsection*{
    \begin{flushleft}
        \Large
        Problem 31: Give up.
    \end{flushleft}
}

%Problem 32
\subsection*{
    \begin{flushleft}
        \Large
        Problem 32:
        [9]
        [Difficulty Estimate=2.3]
        Suppose p is an odd prime and $a, b, c \not\equiv 0 (mod\ p)$.
        Prove that the equation $ax^2 + by^2 + cz^2 \equiv 0 (mod\ p)$
        has at least p solutions $(x, y, z) \in Z_p^3$.
    \end{flushleft}
}%Solution to Problem 32
\begin{flushleft}
    \Large
    Solution: Recall the Example 57 [12]:
    For any prime p, any $a \in Z^\star_p$,
    there exists an integer $b (1 \leq b \leq p - 1)$
    such that the equation
    $x^2 + y^2 + a = bp$
    has an integer solution.\\
    We can see this Example from a new perspective.
    That is to say, $\forall a \in Z_p^\star$,
    a can be displayed as the sum of two quadratic residues.\\
    Even better, we can similarly reach that
    $\forall a \in Z_p^\star$,
    a can be displayed as the sum of two quadratic non-residues.\\
    Armed with this powerful conclution,
    we can deal with the original problem with great ease.\\
    Obviously, $x^2$ is a quadratic residue.
    If a is a quadratic residue,
    then $ax^2$ is still a quadratic residue.
    After fixing a, with x traversing all elements in $Z_p^\star$,
    we can see $ax^2$ traverse all elements in quadratic residue.\\
    If a is not a quadratic residue,
    similarly we can get that
    $ax^2$ traverse all quadratic non-residues.
    And this is the same to $by^2$ and $cz^2$.\\
    Examine the original equation.
    There definitely are two elements,
    which are both quadratic residues or both quadratic non-residues.
    WLOG, they are $ax^2$ and $by^2$, and the equation change to
    $u + v \equiv  w (mod\ p)$.
    w can traverse quadratic residues or quadratic non-residues,
    both of which has $(p-1)/2$ elements.
    And for each w choosed,
    by the conclution mentioned before, there must exists a solution (u,v).
    Since $ax^2$ and $by^2$ both can traverse all elements,
    the solution must can be shown by x, y, i,e, it is valid.
    And (v,u) is also a valid solution, so now for each w,
    we have 2 solutions.
    Summing up them leads to $p - 1$ solutions, while the remaining one is (0,0,0),
    so there are at least p solutions.
\end{flushleft}

%Problem 33
\subsection*{
    \begin{flushleft}
        \Large
        Problem 33:
        (USA TST 2008, Problem 4)
        [Difficulty Estimate=3.2]
        Recall that a perfect square is simply the square of an integer.
        Prove that, for any integer n,
        $n^7 + 7$ is not a perfect square.
        (Hint from [2]: Use Lemma 2.)
    \end{flushleft}
}%Solution to Problem 33
\begin{flushleft}
    \Large
    Solution:
    Prove this by contradiction. Assume that $n^7 + 7 = a^2$.
    Apply mod 4 to each side. $a^2 \equiv 0\ or\ 1\ (mod\ 4)$,
    $n^7 + 7\equiv n^7 + 3 \equiv 3\ or\ 0\ or\ 2 (mod\ 4)$,
    so the only possibility is that
    $n^7 + 7 \equiv a^2 \equiv 0,\ n \equiv 1,\ a \equiv 0\ or\ 2\ (mod\ 4) $.\\
    To draw a contradiction,
    there must be a form that the left hand side can be factorized,
    and the right hand side can be show as two perfect square numbers' sum.
    And that is $n^7 + 2^7 = a^2 + 11^2$,
    where the left hand side can be displayed as
    $(n+2)(n^6 - 2n^5 + 4n^4 - \cdots + 2^6)$.\\
    Use Lemma 2. Since $gcd(a,11)=1$,
    each odd factor of $a^2 + 11^2$ has the form of $4k+1$.
    However, $n+2 \equiv 3,\ n^6 - 2n^5 + 4n^4 - \cdots + 2^6 \equiv 3\ (mod\ 4)$,
    this is the contradiction. Then $n^7 + 7$ is not a perfect square.
\end{flushleft}

%Problem 34
\subsection*{
    \begin{flushleft}
        \Large
        Problem 34:
        [Difficulty Estimate=1.8]
        Suppose $ \{ S_1, S_2 \} $ is a partition of $Z^\star_p$,
        for an odd prime p.
        For any $x, y \in S_1$,
        and any $z, u \in S_2$,
        we always have $xy, zu \in S_1$ and $xz, yu \in S_2$.
        Prove $S_1 = QR_p$ and $S_2 = QNR_p$.
    \end{flushleft}
}%Solution to Problem 34
\begin{flushleft}
    \Large
    Solution:
    We can choose $y = x$, then $x^2 \in S_1$,
    so $S_1$ contains all the quadratic residues,
    and $S_2$ can only contain quadratic non-residue.\\
    Take $z \in S_2$, then $zk^2 \in S_2$ for all k,
    so $S_2$ contains all the quadratic non-residues.\\
    Since all the quadratic residues and quadratic non-residues
    can form the $Z_p^\star$, we get $S_1 = QR_p$ and $S_2 = QNR_p$.
\end{flushleft}

%Problem 35
\subsection*{
    \begin{flushleft}
        \Large
        Problem 35:
        (Spring 2022, Quiz 3-2)
        [Difficulty Estimate=2.5]
        Prove that the equation $4xy - x - y = z^2$
        has no solution in positive integers.
    \end{flushleft}
}%Solution to Problem 35
\begin{flushleft}
    \Large
    Solution:
    $4xy - x - y = z^2\ \Rightarrow \ (4x-1)(4y-1) = (2z)^2 + 1$. \\
    Since $gcd(2z,1) = 1$,
    so all the odd factors of $(2z)^2+1$ has the form of $4k+1$.
    But both $4x-1$ and $4y-1$ don't have that form,
    so this equation cannot hold, i,e,
    it has no solution in positive integers.
\end{flushleft}

%Problem 36
\subsection*{
    \begin{flushleft}
        \Large
        Problem 36:
        (IMO 2020 Shortlist-N2, adpated)
        [Difficulty Estimate=3.2]
        Please prove the following two propositions.
        Feel free to use (1) in the proof of (2).\\
        (1) For any prime p such that $p \equiv 1\ (mod\ 3)$,
        for any $x \in Z^\star_p$,
        either x has three cube roots, or it
        has none.
    \end{flushleft}
}%Solution to Problem 36
\begin{flushleft}
    \Large
    Give up.\\
    Others give up too.\\
\end{flushleft}
\end {document}