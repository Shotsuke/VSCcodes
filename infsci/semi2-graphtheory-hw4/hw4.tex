\documentclass[UTF8]{ctexart}
\usepackage{setspace}
\usepackage{amssymb}
\usepackage[a4paper, total={6.5in, 9.5in}]{geometry}

\begin{document}
%标题
\section*{
  \begin{center}
      \huge
      信息与计算科学导论\\
      Graph Theory\\
      221502023 \  沈硕 \  第四次作业\\
  \end{center}
 }

%Problem 5.4
\subsection*{
    \begin{flushleft}
        \Large
        Problem 5.4:
        Order the vertices of a graph G according to their degrees,
        so that $V(G) = \{x_1,x_2,\cdots,x_n\}$
        and $d(x_1) \geq d(x_2) \geq \cdots$.
        Show that in this order the greedy algorithm uses
        at most $\max_i \min\{ d(x_i) + 1 ,i\}$ colours,
        and so if k is the maximal natural number
        for which $ k \leq d(x_k) + 1$ then $\chi (G) \leq k$.
    \end{flushleft}
}%Solution to Problem 5.4
\begin{flushleft}
    \Large
    Solution:
    Color G by following this non-increasing degree sequence.
    Each time when we color the $i^{th}$ vertex $v_i$,
    it has at most $\min \{d_i , i-1\}$ colored neighbours.
    So at this very moment we use at most $1+\min \{d_i , i-1\}$, or ,
    $\min\{ d(x_i) + 1 ,i\}$ colors.
    Since this holds for all vertices, maximizing i can get the upper bound.\\
    If k is the maximal natural number for which $ k \leq d(x_k) + 1$,
    then coloring $v_1 , v_2 ,\cdots , v_k$ uses at most k colors.
    $\forall l > k\ ,\ l > d(x_l) + 1$,
    then for $v_{k+1} , v_{k+2} , \cdots , v_n$,
    when we come to color each of them, we will use at most $d(x_l) + 1$ colors.
    As the degree sequence is non-increasing, $d(x_l) + 1 \leq k$ holds.
    So $\chi (G) \leq k$.
\end{flushleft}

%Problem 5.5
\subsection*{
    \begin{flushleft}
        \Large
        Problem 5.5:
        Deduce from Exercise 4 that if G has n vertices then
        \begin{center}
            \Large
            $\chi(G) + \chi(\overline{G}) \leq n + 1$
        \end{center}
    \end{flushleft}
}%Solution to Problem 5.5
\begin{flushleft}
    \Large
    Solution:
    We have non-increasing degree sequence for G:
    $V(G) = \{x_1,x_2,\cdots,x_n\}$
    and $d(x_1) \geq d(x_2) \geq \cdots \geq d(x_n)$.\\
    Correpondingly,
    there is a non-increasing sequence for $\overline{G}$:
    $d(x_n) \geq d(x_{n-1}) \geq \cdots \geq d(x_1)$.\\
    Let k be the maximal natural number
    for which $ k \leq d(x_k) + 1$ then $\chi (G) \leq k$.
    Then for $\overline{G}$ this must be $n+1-k$.
    Summing up these two can solve the problem.
\end{flushleft}

%Problem 5.6
\subsection*{
    \begin{flushleft}
        \Large
        Problem 5.6:
        Show that $\chi(G) + \chi(\overline{G}) \geq 2 \sqrt{n}$.
    \end{flushleft}
}%Solution to Problem 5.6
\begin{flushleft}
    \Large
    Solution:
    Since k is at least 1 and at most n,
    we have $\chi(G) \chi(\overline{G}) \geq n$.
    So $(\chi(G) + \chi(\overline{G}))^2 \geq 4\chi(G) \chi(\overline{G})$.
    Then we are done.
\end{flushleft}

%Problem 5.29
\subsection*{
    \begin{flushleft}
        \Large
        Problem 5.29:
        Find the edge chromatic number of $K_n$.
    \end{flushleft}
}%Solution to Problem 5.29
\begin{flushleft}
    \Large
    Solution:
    If n is odd, then $\chi^{'}(G) = n\ (\ = \Delta + 1 )$;
    if n is even, then $\chi^{'}(G) = n-1\ (\ = \Delta )$.\\
    Case even:
    G must contain n-1 different perfect matchings.
    Then for each perfect matching, we apply a different color.
    So their will be $\chi^{'}(G) = n-1\ (\ = \Delta )$.\\
    Case odd: n can be shown as 2k+1, so $\mid E \mid = (2k+1)k$.
    Since each color can color k edges, we need at least
    $\lceil\frac{\mid E \mid}{k}\rceil = 2k+1 = n$.
\end{flushleft}
\end{document}